\documentclass[notumble]{leaflet}

\usepackage{color}
\usepackage{lipsum}
\usepackage{graphicx}

\thispagestyle{plain}
\pagestyle{plain}

\begin{document}

\begin{center}
{\huge\bf How to Help Get \\
Bernie Sanders in the \\
\vspace{10pt}
White House}

\includegraphics[width=\textwidth]{../photos/bernie-b-n-w.png}

{\large\bf Find Bernie's message exciting, but not sure how to help make it a reality?}

\vfill

{\tiny Prepared by Central Ohio Grassroots for Bernie Sanders,\\
  without Sanders' approval.\\
  {\tt http://bit.do/Central-Ohio-Grassroots\\
    so.grassroots@gmail.com}}

\end{center}

\pagebreak

\section*{Bernie Will Only End Up in the White House If We Put Him There Ourselves}

Many people are excited about Bernie Sanders' ``End To Business As
Usual'' progressive campaign for the White House but are not sure how
they can help.  Of course at some point someone is going to ask you to
donate to Bernie's campaign, and I hope then you can give generously
because it will likely be some of the most efficient campaign
money in the race.  But Bernie's communication strategy is not
based on buying TV ads or spectacles for the news to cover.  It is
going to depend much more on genuine human connections between people
like you and the people you meet and talk to about his campaign.

This human connection isn't something you can pay for or generate on a
program.  It starts with the authentic appeal of Bernie's message and
record of public service, proceeds through a series of conversations
between people who might not have known each other before, and
hopefully leads to voluntarily commitments between those people to
organize and help spread his message further. That is how this
pamphlet came to be in your hands today.  This is not a quixotic
endeavor, in it may be the best chance for a Democrat to win the
presidency in 2016.

However, these kinds of authentic relationships take time to develop,
and there is relatively little time left until the presidential
primaries.  Progressives in the US have been atomized to the point
where we hardly know how to find each other any more let alone
organize an effective political campaign.  So if you want to help,
don't wait for organization from the top down -- especially if you're
outside the early battleground states because the official campaign is
focused there for now.  Reach out and find your fellow progressives to
organize together!  Some of the links on the last page might help with
advertising, but what the campaign really needs now is people taking
the lead in their local areas by publically committing to be
physically present at a certain place and time to talk about Bernie.
From local connections like this, you can plan local campaigns to
reach out further through talking to people door-to-door or at
festivals.

\pagebreak

\section*{Bernie's Message}

Bernie's policies are only ``extreme'' to the country's political
elite.  They are common-sense ideas which will improve the lives of
just about everyone.  Some of them are:

\begin{itemize}

\item Raising the minimum wage and expanding social security, which is
  likely to be the most effective economic stimulus possible.

\item Investing \$1T in restoring US infrastructure such as roads and
  bridges.  (Huge jobs program.)

\item Entirely funding public education tuition through a modest tax
  on very large stock transactions.

\item Universal single-payer health care.

\item Ending the secretive and undemocratic negotiation of the TPP
  ``Free Trade'' deal, and developing more and better jobs for people
  in the US.

\item Breaking up enormous financial institutions.  (``If a bank is
  too big to fail, that bank is too big to exist.'')

\item Tax reform ending the sweetheart deals given to the extremely
  wealthy during the Bush \& Obama presidencies.

\item Cracking down on abusive surveillance by US intelligence \& law
  enforcement agencies.

\item Avoiding military conflict in the Middle East.

\item Preventing climate change.

\item Expanding civil rights for minorities, women and LGBT people.

\item Bringing the country back from the brink of oligarchy through
  campaign-finance finance reform which overturns ``Citizen's
  United.''

\end{itemize}

These are not just planks Bernie has adopted as a matter of political
calculation in his bid for the presidency.  All of these policies are
in line with things he's been saying for years, and his record of
public service goes back to the 70s, reflecting a deep concern for the
welfare of the people he represents.  

\pagebreak

\section*{The Challenges Bernie Faces}

The fact is, Bernie's electoral strategy will not work without the
help of millions of people just like you.  He plans to raise around
\$50 million dollars in campaign contributions, a pittance next to
Hillary Clinton's planned \$2 billion, giving her a forty-fold
financial advantage.  Even if he had the most compelling thirty-second
TV ad in political history it wouldn't win him the Democratic
presidential nomination because he won't be able to run it often
enough.

Beyond simple financial disparities, he will also face resistance from
establishment media outlets.  For instance, ABC news led with the
photo below when introducing his campaign, not exactly the most
flattering still they could have taken from his announcement speech.
And Jon Stewart has a good overview of famous pundits disparaging him
as eccentric and irrelevant. (See {\tt
  http://bit.do/bernie-disparaged} for his piece on it.)  One DNC
member even just called Bernie crazy!  Bernie is a bigger threat to
establishment interests than Occupy Wall Street was, and there will be
powerful incentives to discredit and marginalize him.

\begin{center}
\includegraphics[width=0.95\textwidth]{../photos/abc_bernie_sanders_tl_150526_16x9_992-angry-cropped-grayscale-gamma-1point6.jpg}
\end{center}

\pagebreak

\section*{The Strategy}

Around the turn of the Twentieth Century, before Public Relations took
off as a tool for manipulating massive groups of people to act against
their own interests, all politicians campaigned the way Bernie is
expecting to.  They would get out and meet people in person to share
their concerns and to hear others', and they would ask their followers
to do so as well.

This is still a strategy which can work as long as you're a genuine
populist like Bernie is, as shown by his successful history as a
Senator for Vermont.

\subsection*{But Can He Win a National Election?}

As long as enough people really understand his simple message of
returning to democratic governance which responds to the needs of the
governed, he stands a good chance.  There are two aspects to this
question: the general election in November 2016, and the Democratic
presidential nomination in July 2016.

For the general election, {\bf he is actually a stronger candidate as
  the Democratic nominee than Hillary would be}, because through no
fault of her own the Right has been cultivating vitriolic hatred of
her for over a decade.  In a March Pew Center poll, 74\% of
Republicans said there is ``no chance'' they would vote for her.  In
contrast, Bernie has to contend with the fact that his populist
policies can be described as ``Socialist,'' while on the other hand
it's easy to demonstrate that they'll enormously improve most people's
lives because the most important of them are a return to policies seen
as essential at some point in the twentieth century, even by important
Republicans.

For the nomination, Bernie has to win enough delegates in Democratic
state primaries.  Historically speaking, relatively few people vote in
these primaries, but Bernie's populist message has the potential to
bring many more voters out.  For instance, during the last contested
Democratic Ohio presidential primary, in 2008, only 39\% of potential
voters participated in either primary.  Turnout for the last Ohio
Senate race was similar.  In contrast 59\% of potential Vermont voters
turned out for Bernie's last Senate race.

\pagebreak

\section*{Ohio-specific information}

There is a lot to do to help Bernie in Ohio.  It doesn't matter how
red Ohio ends up going in the general election, in fact even if
another person ``wins'' the Democratic Ohio primary, efforts to spread
Bernie's message here will not go to waste.  This is because the
primary election is not winner-take-all.  Based on the primary, Ohio
will send 148 delegates to the Democratic National Convention in July
2016 who are pledged to vote for the various candidates for the
presidential nomination, and the number pledged to Bernie will be
roughly proportional to the number of votes he gets in Ohio.  {\bf
  Every primary vote you win for Bernie will bring him closer to the
  White House.}

There is a plan afoot to use the Ohio voter registration rolls and
precint-level data from the 2008 primary to identify precincts which
went strongly for Obama but had relatively few participants given
their size.  We (the people who put this brochure together) believe
that these precincts will be good targets for talking to people
door-to-door, shepherding voter registrations through the electoral
board, and setting up absentee ballots.  We are also planning to
talk to people at festivals in Central Ohio.  

Those are the ideas we are working with, but we'd love to hear more,
and even more, we'd love to hear about other groups starting up!
(Though of course you're also welcome at ours!  And if you have any
skill with GIS and data analysis we could use some help with the
precinct analysis.)

We've found the following online groups useful in advertising our
events:

\begin{itemize}

\item {\tt http://www.peopleforbernie.com/users/\\
  event\_pages/new?parent\_id=6}
\item {\tt https://reddit.com/r/ohioforsanders/}
\item {\tt https://www.facebook.com/OhioForSanders}

\end{itemize}

National groups which are also worth connecting with:

\begin{itemize}
\item {\tt http://reddit.com/r/sandersforpresident}
\item {\tt https://www.facebook.com/groups/\\
  786752971370959/}

\end{itemize}

\end{document}

% https://drive.google.com/file/d/0B3qaT-ZL6aeKSE1NcFZrVTFXM2c
